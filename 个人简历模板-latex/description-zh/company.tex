\resheading{科研项目}
  \begin{itemize}[leftmargin=*]

      \item
      \ressubsingleline{校优毕设“面向配网设备的可重构数据采集系统设计”}{\textbf{排名:$\,${\textbf{}No.1/203(推荐省优)}  \quad}}{\textbf{2021.05 - 2023.05}}
      {\small
      	\begin{itemize}
      	\item 面向配电网设备,针对温度、气压、电流、电压、暂态地放电以及超声波监测等传感器的数据采集需求,设计并实现了一种\textbf{配网设备数据采集器硬件}(PCB)。
            \item 引入系统可重构的概念到配网设备数据采集系统,\textbf{提出一种基于软件的可重构实现方案}。设计并开发了基于QT的复杂重构软件与基于USART-HMI的现场配置软件,二者相互配合,提高了系统的灵活性和适应性。
            \item 基于测试平台,结合指标要求,对各模块功能进行\textbf{调试和结果分析}。实验结果显示,本文所设计的面向配网设备的可重构数据采集系统在满足不同场景下的数据采集需求的同时,具有较高的可靠性和实用性。
      	\end{itemize}
      }


      \item
      \ressubsingleline{国家级大创“新一代人工智能驱动的矿工违规操作行为诊断系统”}{\textbf{负责人}}{\textbf{2021.05 - 2023.05}}
      {\small
      	\begin{itemize}
      	\item 基于非逆方法\textbf{改进随机配置神经网络(SCNs)}输出权值更新机制及论文、专利撰写。
            \item 基于传感器的\textbf{人体行为模态识别方法}研究及MATLAB编程实现。
            \item 多场景矿工违规行为诊断\textbf{机械装置设计}、专利撰写。
            \item 人体行为模态\textbf{数据}快速\textbf{获取}、\textbf{特征工程}处理及基于QT的\textbf{矿工违规行为诊断软件}设计与研发。
      	\end{itemize}
      }
  
      \item
  \ressubsingleline{“基于信息物理系统架构的重介质选煤硬件在环仿真系统”项目研究}{\textbf{负责人}}{\textbf{2020.04 - 2021.04}}
  {\small
  	\begin{itemize}
  		\item 基于Unity3D 的重介质选煤虚拟信息系统设计与基于Simulink 代码生成的实时物理控制器设计,项目获\textbf{第六届全国高等学校矿物加工工程专业学生实践作品大赛三等奖}。
  	\end{itemize}
  }
  
  \item
    \ressubsingleline{国家级大创“基于卡口数据的交通观测点相关性分析”}{\textbf{排名第3}}{\textbf{2021.05 - 2023.05}}
  {\small
  	\begin{itemize}
  		\item 通过融合节点相似性和交通流量对PageRank算法进行改进以及相关专利撰写。
  	\end{itemize}
  }


  
  \end{itemize}



% \resheading{实习经历}
%   \begin{itemize}[leftmargin=*]
  
%       \item
%   \ressubsingleline{\textbf{美敦力医疗中国研发中心-上海闵行}}{Lab\&RCS}{\textbf{2019.12 - 2021.03}}
%   {\small
%   	\begin{itemize}
%   		\item 快速适应新环境,并将在校习得理论与公司测试实验实际相结合,参与了\textbf{肾透析}和\textbf{喉镜}产品研发过程中各部分的测试,包括机械、化学、可靠性实验等。
%   	\end{itemize}
%   }
%   \end{itemize}
